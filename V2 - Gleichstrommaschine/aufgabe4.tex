\chapter{}
In dieser Aufgabe soll bei unseren fremderregten Gleichstrommaschine die Parameter $ c_{E}\Psi_{N} $ und $ R_{A} $  ermittelt werden. Der Ankerstrom IA, die Ankerspannung UA und die Drehzahl N stehen uns dabei als Messgrößen zur Verfügung.
\section{}
In der ersten Aufgabenstellung soll ein Verfahren bestimmt werden, wie man $ c_{E}\Psi_{N} $ und $ R_{A} $ ermitteln kann. Dabei sind die Betriebszustände der Maschine zu beachten. Um $ c_{E}\Psi_{N} $ und $ R_{A} $ auszurechnen bedienen wir uns den zwei Gleichungen (\ref{eq:4:NN0}) und (\ref{eq:4:cepsi}).
\begin{equation}
	N_{N0} = \frac{U_{AN}}{c_{E}\psi_{N}}
	\label{eq:4:NN0}
\end{equation}
\begin{equation}
	c_{E}\psi_{N} = \frac{U_{AN}}{N_{N0}}
	\label{eq:4:cepsi}
\end{equation}

\begin{equation}
	U_{A} = R_{A}I_{A} + U_{i}~~ mit~ U_{i} = c_{E}\psi_{N}
	\label{eq:4:ua}
\end{equation}
daraus ergibt sich der Ankerwiderstand:
\begin{equation}
	R_{A} = \frac{(U_{A} - c_{E}\psi_{N})}{I_{A}}
	\label{eq:4:ra}
\end{equation}
Um $ c_{E}\Psi_{N} $ zu ermitteln wenden wir die nun Gleichung (\ref{eq:4:cepsi}) an. Man muss aber beachten, das der Ankerstrom hier 0 sein muss, damit $ U_{A} = U_{i} $ (2.1) gilt. Dies erreichen wir, indem wir unsere Gleichstrommaschine mit dem Synchronmotor als Lastmaschine antreiben und somit unseren GM auf eine konstante Drehzahl N setzen. Um $ R_{A} $ zu bestimmen wird die GM im stationären Betrieb betrieben, weil wir hier den Ankerstrom benötigen überhaupt die Formel (\ref{eq:4:ra}) anwenden zu können.


\section{}
In dieser Teilaufgabe sind nun die Parameter $ c_{E}\Psi_{N} $ und $ R_{A} $ durch unsere Annahmen im Aufgabenteil a) zu bestimmen.\\
Für unseren zu berechnetem Parameter $ c_{E}\Psi_{N} $ haben wir unseren GM, mittels den Betrieb des Synchronmotors, auf eine konstante Drehzahl N versetzt. Die Ankerspannung wurde mithilfe eines Multimeters direkt gemessen. Weil die Messungen ungenau haben wir den Mittelwert von drei unterschiedliche Messgrößen gebildet. Die Tabelle \ref{tab:4b:cepsi} stellt unsere Messwerte mit den errechneten $ c_{E}\Psi_{N} $ dar.

\begin{table}[h]
	\centering
	\begin{tabular}{p{1.5cm} p{1.5cm} p{1.5cm} | p{1.5cm}}
		&&&\\[-1em]
		$ U_{A}/V $ & $ N/min^{-1} $ & $ N/s^{-1} $ &  $ c_{E}\Psi_{N}/Vs $\\
		\hline &&&\\[-1em]
		73 & 1500 & 25 & 2.9200 \\
		87.5 & 1800 & 30 & 2.9167 \\
		77.9 & 1600 & 26.7 & 2.9213
	\end{tabular}
	\caption{Messungen von $ U_{A} $ und $ N $ zur Bestimmung von $ c_{E}\Psi_{N} $}
	\label{tab:4b:cepsi}
\end{table}
Der Mittelwert von $ c_{E}\Psi_{N} $ aus Tabelle \ref{tab:4b:cepsi} ergibt hier:
\begin{equation}
	c_{E}\psi_{N} = \frac{1}{3} \sum\nolimits_{z=0}^3 \frac{U_{AN}}{N_{N0}} = 2.919Vs
\end{equation}

Bei dem zu berechneten Ankerwiderstand haben wir unseren GM im stationären Betrieb betrieben. Der Ankerstrom $ I_{A} $ und die Drehzahl $ N $ konnten wir direkt auf unserer Bedienoberfläche ablesen. Die einstellbare Ankerspannung $ U_{A} $ haben wir auch hier auf mehrere Werte eingestellt und mit den wert der ermittelten Maschinenkonstante aus der Tabelle \ref{tab:4b:cepsi} in die Formel (\ref{eq:4:ra}) eingesetzt und diese alles in der Tabelle \ref{tab:4b:ra} darstellt.
\begin{table}[h]
	\centering
	\begin{tabular}{p{1.5cm} p{1.5cm} p{1.5cm} p{1.5cm} | p{1.5cm}}
		&&&&\\[-1em]
		$ I_{A} $ & $ U_{A}/V $ & $ N/min^{-1} $ & $ N/s^{-1} $ &  $ R_{A}/\Omega $\\
		\hline &&&&\\[-1em]
		2.65 & 100 & 1705 & 28.4 & 6.43 \\
		2.70 & 110 & 1908 & 31.8 & 6.36 \\
		2.74 & 120 & 2106 & 35.1 & 6.40
	\end{tabular}
	\caption{Messungen von $ I_{A} $, $ U_{A} $ und $ N $ zur Bestimmung von $ R_{A} $}
	\label{tab:4b:ra}
\end{table}
Dabei ergibt der Mittelwert von $ R_{A} $ aus Tabelle \ref{tab:4b:ra}:
\begin{equation}
	R_{A} = \frac{U_{A} - c_{E}\psi N}{\frac{1}{3} \sum\nolimits_{z=0}^3 I_{A}} = 6.40\Omega
\end{equation}

