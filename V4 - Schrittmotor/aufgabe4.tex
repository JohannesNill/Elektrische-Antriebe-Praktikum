\chapter{}\label{ch:aufg4}

\section{}\label{sec:4a}
Es soll nun der Schrittwinkel $ \alpha $ im Voll- und Halbschrittbetrieb des Schrittmotors bestimmt werden. Dafür werden die folgenden Formeln \ref{eq:4a:vs} und \ref{eq:4a:hs} herangezogen. Gegeben sind die Daten Polpaarzahl und Strangzahl: 
\begin{center}
	$ Z_{P} = 1 $ \hspace{2cm} $ m = 2 $ 
\end{center}

\begin{equation}
	\alpha_{VS}[Z_{P}=2] = \frac{360^{\circ}}{kZ_{p}m} = \frac{360^{\circ}}{2*1*2} = 90^{\circ}
	\label{eq:4a:vs}
\end{equation}
\begin{equation}
	\alpha_{HS}[Z_{P}=2] = \frac{360^{\circ}}{kZ_{p}m} = \frac{360^{\circ}}{4*1*2} = 45^{\circ}
	\label{eq:4a:hs}
\end{equation}

\section{}\label{sec:4b}
Berechnet werden soll nun jeweils der Schrittwinkel $ \alpha $, sowie die Schrittzahl pro Umdrehung z. Dies soll sowohl für den Halbschrittbetrieb, als auch den Vollschrittbetrieb durchgeführt werden. Gegeben ist die Polpaarzahl $ Z_{P} = 50 $, sowie die Strangzahl $ m = 2 $.
Nach den obig in a) gegebenen Formeln \ref{eq:4a:vs} und \ref{eq:4a:hs} berechnen wir nun die Schrittwinkel im Voll- und Halbschrittbetrieb. Hierbei berücksichtigen wir natürlich die im Aufgabenteil b) gegebenen Werte.
\begin{equation}
\alpha_{VS}[Z_{P}=50] = \frac{360^{\circ}}{kZ_{p}m} = \frac{360^{\circ}}{2*50*2} = 1.8^{\circ}
\label{eq:4b:vs}
\end{equation}
\begin{equation}
	\alpha_{HS}[Z_{P}=50] = \frac{360^{\circ}}{kZ_{p}m} = \frac{360^{\circ}}{4*50*2} = 0.9^{\circ}
	\label{eq:4b:hs}
\end{equation}

Um nun an die Schrittzahl des Motors zu gelangen, müssen wir $ 360^{\circ} $ (eine volle Umdrehung), durch den jeweilig ermittelten Schrittwinkel teilen. So erhalten wir:
\begin{center}
	Schrittzahl $ VS = 200~Schritte $\\
	Schrittzahl $ HS = 400~Schritte$
\end{center}

\section{}\label{sec:4c}
Es soll ermittelt werden, wie viele Schritte der Schrittmotor im Vollschritt-, bzw. Halbschrittbetrieb ausführen muss, damit der Schlitten 24mm verfahren werden kann. Die Spindelsteigung der Kugelgewindespindel beträgt 20mm pro Umdrehung.\\
Zuerst wird ermittelt, wie viele Umdrehungen bzw. um wie viel Grad der Motor drehen muss, damit die 24mm erreicht werden:
\begin{equation}
\alpha[l = 24mm] = \frac{24mm}{20mm}360^{\circ} = 432^{\circ}
\label{eq:4c:grad}
\end{equation}
Der hier erhaltene Winkel wird in die Formeln nun durch $ \alpha_{VS} $ bzw. $ \alpha_{HS} $ geteilt, wodurch sich die benötigten Schrittzahlen für den Verfahrweg ergeben:
\begin{center}
	Schrittzahl $ VS = \frac{432^{\circ}}{1.8^{\circ}} = 240~Schritte $\\
	Schrittzahl $ HS = \frac{432^{\circ}}{0.9^{\circ}} = 480~Schritte$ 
\end{center}
