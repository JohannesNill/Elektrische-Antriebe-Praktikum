\chapter{}\label{ch:auf6}
\section{}\label{sec:6a}
Im Folgenden soll der Unterschied zu den tatsächlich gemessenen Stromverläufen dargelegt werden.\\ 
Im Labor hat sich herausgestellt, dass es einen Unterschied zwischen den idealisierten und den tatsächlichen Stromverläufen gibt. Auf dem Oszilloskop ist zu erkennen, dass wenn W eingeschaltet, V ausgeschaltet wird, es zu einem Einbruch des Stromes U kommt. 
Der Einbruch des Stromes erfolgt um den Betrag des Schnittpunktes der ein- bzw. ausgeschalteten Phasenströme. \\
Begründen lässt sich dieser Effekt dadurch, dass die Wicklungen in Sternschaltung zu einem gemeinsamen Knoten zusammengeführt sind. Es findet eine gegenseitige Beeinflussung der Ströme beim Umschalten zwischen den Spulen statt. Der Auf- bzw. Abbau der Magnetfelder benötigt eine gewisse Zeit. Dies kann zu derartigen Störungen führen. Ebenfalls ist zu bemerken, dass die Ströme nicht sprunghaft, sondern mit PT-1 Verhalten ansteigen, da es sich um Induktivitäten handelt.

\section{}\label{sec:6b}
\textit{Welche Auswirkungen haben diese Stromverläufe auf das Drehmoment?}\\
Die nicht idealisierten Stromverläufe des Ankerstroms weisen kurze Einbrüche auf. Diese verursachen einen kurzzeitigen Drehmomentverlust.

\section{}\label{sec:6c}
\textit{Mit welcher Maßnahme lässt sich dieses Verhalten verbessern?}\\
Damit die Stromläufe verbessert werden können, kann man auf eine Kombination aus einem Stromregler und einer Drehzahlregelung zurückgreifen. Dabei wird die Drehzahlregelung dazu benötigt, die Selbstzerstörung der Maschine bei konstantem Strom zu verhindern.
